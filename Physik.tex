% .:: Laden der LaTeX4EI Formelsammlungsvorlage
\documentclass[fs, footer]{latex4ei}
\usepackage[european]{circuitikz}

\usepackage{multirow}
\usepackage{latexnew}


% Dokumentbeginn
% ======================================================================
\begin{document}


% Aufteilung in Spalten
\vspace{-4mm}
\begin{multicols*}{4}
	\fstitle{Physik}


	\emphbox{
	\textbf{Wichtiger Hinweis}
	\\ Diese Formelsammlung ist noch in der Entwicklung und nicht prüfungstauglich ! \\ Allerdings würden wir uns über Unterstützung freuen das zu ändern. Wer Lust hat kann uns über das Kontaktformular auf www.latex4ei.de erreichen.
	}
% ===============================================================================================
\section{Messgenauigkeit und Messfehler}
Systematischer Fehler: Abweichung einer Messung von ihrem Erwartungswert\\
Statistischer Fehler: Entstehung durch zufällige positive bzw. negative Abweichungen\\
Arithmetischer Mittelwert: $\ol x = \fr{1}{n}\sum_{i=1}^n x_i$\\
Standartabweichung: $s = \sigma = \sqrt{\fr{1}{n-1}\sum_{i=1}^n (x_i-\ol x)^2}$\\
Normalverteilung/Gauß-Funktion: $g(x) = \fr{1}{\sigma \sqrt{2\pi}\exp(-\fr{(x-\ol x)^2}{2\sigma^2})}$\\
Näherungsweise gilt: 
\begin{itemize}
\item 68\% aller Messwerte haben eine Abweichung < $\pm \sigma$ vom Mittelwert.
\item 95\% aller Messwerte haben eine Abweichung < $\pm 2\sigma$ vom Mittelwert.
\item 99,8\% aller Messwerte haben eine Abweichung < $\pm 3\sigma$ vom Mittelwert.
\end{itemize}
%TODO Fehlerfortpflanzung

\section{Konstanten}
$ \epsilon_0 = 8.85\cdot 10^{12} \fr{C^2}{Nm^2}$\\

\section{Kinetik}
momentane Geschwindigkeit: $v = \dot r$\\
mittlere Geschwindigkeit: $v_m = \fr{\Delta r}{\Delta t}$\\
\subsection{Galilei Transformation}
Gilt nur für $v<<c$\\
$x' = x - ut$ und $t' = t$ mit der Geschwindigkeit $u$ des bewegten Systems $\ra \dt x = \dt{x'} + u$\\
\subsection{Schiefe Ebene}
Gewichtskraft: $F_G = mg$\\
Normalkraft: $F_N = mg\cos \alpha$\\
Hangabtriebskraft: $F_H = F_A = mg\sin \alpha$\\
Reibung: Körper steht, falls $F_{\text{Haft}} = F_{\text{Hang}}$\\
kritischer Neigungswinkel: $tan \alpha = \mu_h$\\

\subsection{Eindimensionale Bewegungen}
Mittlere Beschleunigung: $a = \dt v$\\
Gleichförmige, geradlinige Bewegung: $x(t) = v_0t+c$\\
Gleichförmig beschleunigte Bewegung: $x(t) = \fr{1}{2}a_0t^2+v_0t+x_0$\\
\subsubsection{Konstante Geschwindigkeit}
Momentane Geschwindigkeit: $v = \dt r$\\
\subsection{Zweidimensionale Bewegungen}
Unabhängige Bewegungen in den einzelnen Raumrichtungen\\
Schiefer Wurf:
Berechnung von z(x) durch Eliminieren von t: \\$x(t) = v_{0x}t \Rightarrow t = \fr{x}{v_{0x}}$\\
$z(x) = -\fr{1}{2}g(\fr{{x}}{v_{0x}})^2 + \fr{v_{0z}}{v_{0x}}x = -\fr{g}{2v^2_{x0}}x^2 + tan\theta x$\\
\subsection{Kreisbewegung}
Winkel $\phi = \fr{s}{r}$, 	Bogenlänge $s$, Radius $r$, Umlaufdauer $T$\\
Kreisfrequenz $\omega = \dt \phi =\fr{2\pi}{f}$,	Frequenz $f$\\
Krummlinige Bewegung: $\ \v a = \dt {\v v} = \v a_t + \v a_{zp}$,	Tangetialbeschl. $a_t$\\
\subsection{Pendel}
$\omega = \sqrt{\fr{g}{l}}$\\
\subsection{Stöße}
Impuls: $p = mv$, $F=\dot p$\\
\subsubsection{Inelastischer Stoß}
Massen bilden gemeinsame Masse: $v_1' = v_2' = v'$\\
\subsubsection{Elastischer Stoß}
Fall $m_1 = m_2$: $v_1' = v_2, v_2' = v_1$\\
Fall $m_1 = m_2, v_1 \neq 0, v_2 = 0$: $v_1' = 0, v_2' = v_1$\\
$\v v_{\text{1,end}} = \frac{1}{m_1+m_2}{(m_1-m_2)\v v_{\text{1,anf}} + 2m_1\v v_{\text{2,anf}}}$\\


\section{Kraft}
%TODO Eventuell Newtonsche Gesetze, mt =/= const VL4
Kräfte werden vektoriell addiert: $\v F_{ges} = \sum_{i=1}^n \v F_i$\\
Gravitationskraft: $F_G = -G \fr{m_1m_2}{r_{12}^2}$, mit $G = 6,67\cdot 10^{-11} \fr{Nm^2}{kg}$\\
Zentripetalkraft: $F_Z = m \fr{v^2}{r} = m\omega r$\\
Federkraft (Hooke'sches Gesetz): $\v F_F = -k\v x$\\
mittlere Kraft: $\abs{<\v F>} = \abs{\fr{\Delta p}{\Delta t}} = \abs{\fr{m(v_E - v_A)}{\Delta t}}$\\
Coulombkraft: $F = \fr{1}{4 \pi \epsilon_0}\fr{Q_1Q_2}{r^2}$\\
Reibungskräfte allgemein: $\v F_R = \mu \v F$	,z.B. Haft-, Gleit- und Rollreibung\\
Körper beginnt zu rutschen, wenn $\mu_H >= tan \theta$\\
Luftwiderstand: $\v F_W = \frac{1}{2}\rho c_WAv^2$ mit $\rho$: Luftdichte\\

%TODO Evtl subsection Scheinkräfte VL5

\section{Arbeit}
Generell: $W = \int_{r_1}^{r_2} F dr$ bzw. $W = Fs\cos \alpha$\\
Spannarbeit an einer Feder: $W = \fr{1}{2} k(x-x_a)^2 $\\

\section{Energie}
potentielle Energie: $E_{pot} = mgh$\\
kinetische Energie: $E_{kin} = \fr{1}{2}mv^2$\\
\subsection{Energieerhaltung}
Grundprinzip: $E_{vorher} = E_{nachher}$\\
Gesamte Rotationsenergie: $E_{rot} = (\sum_{i=1}^N \fr{1}{2} \Delta m_ir_{i\perp}^2\omega ^2$\\

\section{Leistung}
$P = \dt W = \v F\v V = \dt E$\\



%TODO Massenmittelpunkt VL8

\section{Galilei-Transformation}
Transformation erleichtert Bezugssystem mit konstanter Geschwindigkeit
$\rightarrow$ Berechnung im Schwerpunktsystem

\section{Drehungen}
Drehmoment: $\v M = \v r \times \v F$\\
Drehimpuls: $\v L = \v r \times \v p$\\
%TODO \rightarrow \v L = mr^2 \v \omega \rightarrow \v L = J \v \omega $\\
Trägheitsmoment: $J = \sum_{i=1}^n m_iR_i^2 = \int_V r_\perp^2\rho dV$\\
%TODO VL 8 S21 Tabelle Trägheitsmomente
Satz von Steiner: $J = J_S + Md^2$ Bei bel. Achse A: Summe vom $J_S$ der Rotation durch Schwerpunkt + $Md^2$ von Schwerpunkt um A\\ %TODO BESSER
$E_{kin}(\Delta m_i)=\frac{1}{2}\Delta m_iv_i^2=\frac{1}{2}\Delta m_ir_{i\perp}^2 \omega^2$\\
Gesamte Rotationsenergie: $E_{\text{rot}}=\lim_{N \rightarrow \infty} (\sum_i^N  \fr{1}{2}\Delta m_ir_{i\perp}^2 \omega^2)=\frac{1}{2} \omega ^2\int_Vr_\perp^2 dm$\\
Für ein Teichensystem: $J = \sum_{i}m_ir_{i\perp}^2 \Rightarrow E_{\text{rot}}=\fr{1}{2}J\omega^2$\\
\subsubsection{Trägheitsmomente}
Vollzylinder: $J = \fr{1}{2}h\pi \rho R^4$\\
Hohlzylinder: $J = \fr{1}{2}h\pi \rho[R^4-(R-d)^4] \approx 2 \pi h\rho R^3d$\\
Masse des Mantels: $M \approx 2 \pi R h d \rho$\\
Energieerhalt. rollender Zylinder: $E_{\text{pot}} = E_{\text{kin,translation}} + E_{\text{rotation}} \rightarrow mgh = \fr{1}{2}mv_s^2 + \fr{1}{2}J\omega ^2; s = r\alpha, v = r\omega$\\
%TODO Tabelle Trägheitsmomente Demtröder VL9 S11
%TODO evtl analogien Translation Rotation VL10

\section{Planetenbewegung}
1. Keplersches Gesetz: Planetenbahnen sind Ellipsen um Sonne in einem der beiden Brennpunkte\\
2. Keplersches Gesetz: In gleicher Zeit wird die gleiche Flaeche an einer Bahn aufgespannt\\$\frac{dA}{dt} = \frac{1}{2} \v r \v v sin \alpha = \frac{1}{2}\v r \times \v v = fr{1}{2m}|\v L| \Rightarrow$ Drehimpuls ist zeitlich konstant\\
3. Keplersches Gesetz: $\frac{T_1^2}{T_2^2} = \frac{a_1^3}{a_2^3}$\\
%TODO evtl Spezialfall Kreisbahn

\section{Schwingungen}
\subsection{Harmonische Schwingungen} $x(t) = A \cos(\omega_0t + \Phi)$\\
A = Amplitude; $\omega$ = Kreisfrequenz [rad/s]; f = Frequenz[1/s]; T = Schwingungsdauer 1/f; $\phi$ = Phasenkonstante;\\
$\omega ^2 = \frac{\text{Rücktreibende Kraft}}{\text{Einheitsmasse $\times$ Einheitsauslenkung}} = \frac{k}{m};	\omega = 2 \pi f \rightarrow f = \frac{1}{2\pi}\sqrt{\frac{k}{m}}$\\
Energiebilanz: $E_{\text{ges}} = E_{\text{pot}} + E_{\text{kin}} = \frac{1}{2}kx^2+\frac{1}{2}mv^2$\\
\subsection{Mathematisches Pendel} $F = -mg \sin \theta \approx -mg \theta$ %TODO : Bis 15$°:$ Fehler $<1%$\\
$x = l\theta ; F=-\frac{mg}{l}x$\\
Hooke'sches Gesetz: Kraft proportional zur Auslenkung\\
\subsection{Torsionsschwingungen}
Elastisches Rückstelldrehmoment $M = -D\theta = J\alpha$\\
$\alpha = \frac{d^2 \theta}{dt^2}; \frac{d^2\theta}{dt^2}+\frac{D}{J}\theta	= 0; \omega = \sqrt{\frac{D}{J}}$\\
\subsection{Gedämpfter harmonischer Oszillator}
Stoke'sche Reibungskraft: $F_R = -bv = -bx$\\
Bewegungsgleichung: $\ddot{x} + 2\gamma \dot{x} + \omega_0^2x = 0$; mit 2$\gamma$ = $\frac{b}{m}$\\
Lösungsansatz mit Cosinus: $x = Ae^{-\gamma t} \cos(\omega 't)$ \\mit $\omega ' = \sqrt{\omega_0^2-\gamma^2}$\\
schwache Dämpfung: $x = Ae^{-\gamma \frac{-t}{t_L}} \cos(\omega 't); \gamma = \frac{b}{2m}; \omega_0 = \sqrt{\frac{k}{m}}$\\
aperiodischer Grenzfall: $\gamma = \omega_0 \rightarrow	\omega' = 0$\\
überkritische Dämpfung: $\omega \ll \omega_0 \rightarrow \omega' = \sqrt{\omega_0^2 - \gamma^2}$ = img. $\omega'$ wird imaginär. Das System schwingt nicht, kehrt langsam in GGP zurück\\ 
$t_L$ = mittlere Lebensdauer, Zeit auf 1/e der Amplitude\\
\\Überlagerung von Schwingungen: $x(t)=\sum_nx_n(t)=\sum_na_n\cos{\omega_nt+\delta_n}$\\
%TODO evtl Log Dekrement VL 11, mehr zu Diffgl und erzw Schwing

\section{Hydromechanik}
ideale Gasgleichung: $ \fr{\rho _0}{P_0} = \fr{M}{RT}$\\
Luftdruck:

\section{Wellen}
Polarisation in Materie: $\v P = \chi_e\varepsilon_0\v E$ $\chi_e:$ Elektrische Suszeptibilität, Materialeigenschaft, allg komplex\\
Longitudinale Welle: Auslenkung in Ausbreitungsrichtung\\
Transversale Welle: Auslenkung normal zur Ausbreitungsrichtung\\
Geschwindigkeit Seilwelle: $v = \sqrt{\frac{F_T}{\mu}}$\\ %TODO evtl S11 VL11 mit Bild
%TODO Formeln Geschwindigkeit VL11
Elastizitätsmodul: $E = \frac{F / A}{\Delta l / l}$\\
Kompressionsmodul: $K = \frac{-p}{\Delta V / V}$\\
Ausbreitungsgeschwindigkeit Transversaal $v = \sqrt{\frac{F}{\mu}}$\\
Ausbreitungsgeschwindigkeit Longitudinal $v_\text{l} = \sqrt{\frac{E}{\rho}}$\\
Ausbreitungsgeschwindigkeit in Gasen: $v_\text{l,Gas} =  = \sqrt{\frac{K}{\rho}}$\\
Schwingungsenergie des Teilchens: $E = \frac{1}{2}kD_M^2$\\
$k = 4\pi ^2mf^2; E = 2\pi^2mf^2D_M^2$\\
$m = \rho V = \rho A v t; \Delta E = \Delta E = 2\pi^2 \rho A v \Delta tf^2D_M^2$\\
%TODO v durch kappa oder was auch immer ersetzen
Durchschnittliche Leistung: $P (Strich) = fr{\Delta E}{\Delta t} = 2\pi ^2 \rho a v f^2 D_M^2$\\

Intensität: $ I = \fr{P (Strich)}{A} = 2\pi^2\rho v f^2D_M^2$\\
Intensität sphärische Welle: $I = \frac{P (dach)}{r\pi r^2}$\\
$D_M \propto \frac{1}{r}$\\
%TODO Mathematische Beschreibung der Wellenausbreitung, Wellengleichung VL12
Schallpegel = $L = 10\log{\frac{I}{I_0}}$ mit $I_0 = 10^{-12}\frac{W}{m^2}$\\
Einheit db: 1 Bel = 10db\\
%TODO evtl Schallpegeltabelle
%TODO Superpositionsprinzip VL13
\\
Reflexion bei elektrischen Leitungen: $ r = \frac{Z_L-Z_0}{Z_L+Z_0}$\\
%TODO subsection Interferenz VL13, Überlagerung von Wellen unterschiedlicher Frequenz

\subsection{Geometrische Optik}
$f\cdot\lambda = c$\\
$c_0=2,99792458\cdot 10^8 \frac{m}{s} = \frac{1}{srt{\epsilon_0 \mu_0}}$\\
%TODO was ist Licht?
Energie Photonen: $h\cdot v$ mit $h=6.626\cdot10^{-34}\frac{J}{s}$\\
Brechungsindex $n = \frac{c}{v}$\\
Dielektrizitätskonstante $\varepsilon = n^2 $\\
 Brechungsgesetz von Snellius: $\frac{\sin\theta_1}{\sin\theta_2} = \frac{v_1}{v_2}  = \frac{c/n_1}{c/n_2} = \frac{n_2}{n_1}$\\
Snellius'sche Gesetz für bestimmte Winkel: $n_1\sin\theta 1 = n_2\sin\theta 2$\\
Licht bricht immer zum Medium mit dem höheren Index hin\\
Fermatsches Prinzip:\\
Licht folgt dem Weg mit der kürzesten Laufzeit$: \frac{dt}{dx} = 0$\\
%TODO evtl Fata Morgana
Optische Wand, parallelverschiebung um $\Delta d: d = t\cdot\sin(\alpha)\cdot[1-\frac{\cos(\alpha}{\sqrt{n^2-sin^2(\alpha}}]$\\ %TODO große Klammer
Totalreflexion: falls $\theta>\theta_g$: $\sin(\theta_g) = \frac{n2}{n1}$\\
%TODO evtl Propagationsverluste in der Glasfaser oder was auch immer
%TODO Bild vom Prisma einfügen VL15 S20
\\Brechungsindex n ist frequenzabhängig $n = n(\omega)$\\
Dispersion: $v(\omega)$\\
Maxwell Relation: $n = \sqrt{\varepsilon_r} = \sqrt{1+\chi_E}$\\
Elektrische Suszeptibilität: $\v P = N \cdot \v p$\\% %TODO war das nicht schon oben? N im text
%TODO Bewegungsgleichung erzwungene Schwingung VL16 S7
$x_0 = \frac{eE_0}{m(\omega_0^2-\omega^2}$\\
$\omega < \omega_0$: Auslenkung in Phase,
$\omega > \omega_0$: Auslenkung gegen Phase $F_{el}$\\
Dipolmoment: $|\v p(t)| = e\cdot x(t) = | \frac{e^2E_0\cdot sin(\omega t)}{m(\omega_0^2-}\omega^2|$\\ %TODO große Betragsstriche
$\chi _e(\omega) = \frac{Ne^2}{\varepsilon_0(\omega_0^2 - \omega^2)}$\\
Sellmeier Gleichung: $n^2(\lambda) = 1 + \frac{B_1\lambda^2}{\lambda^2-C_1}\frac{B_2\lambda^2}{\lambda^2-C_2}\frac{B_2\lambda^2}{\lambda^2-C_2}$\\ mit $B_i und C_i (i \in 1-3)$ Sellmeier Koeffizienten, experimentell ermittelt\\
Anormale Dispersion: n steigt mit $\lambda$\\
Normale Dispersion: n fällt mit $\lambda$\\
\subsection{Abbildung}
Entweder reales Bild oder virtuelles Bild (z.B. Spiegel)\\
%TODO Spiegelarten und Strahlengang evtl mit Bildern
Strahlenkonstruktion allgemein: $\frac{1}{b}= \frac{1}{f} - \frac{1}{g}$\\
fokale Länge f = $\frac{r}{2}$; Gegenstandsweite g; Bildweite b\\
Vorzeichen korrekt wählen %TODO Tabelle VL16 S18
%TODO Linsenarten mit Bildern evtl
Abbildungsmaßstab $V = \frac{B}{G}=\frac{-b}{g}$ mit V negativ: Bild umgekehrt\\
\subsubsection{Linsen}
Linsengleichung:
Gegenstandseite: $\frac{f}{g} = \frac{B}{B+G}$
Bildseite: $\frac{f}{b}= \frac{G}{G+B}$\\
Dünne vs dicke Linsen
Reziproke Brennweite = Brechkraft $\rightarrow$ Einheit Dioptrie [D]= 1dpt = $\frac{1}{m}$\\
$g>f$: Reelles Bild;
$g<f$: Virtuelles Bild\\
Berechnung Brennweite: $\frac{1}{f}=(n-1)(\frac{1}{r_1}-\frac{1}{r_2})$\\

\subsubsection{Le Auge}
Weitsichtigkeit: Bild naher Gegenstände hinter Netzhaut $\rightarrow$ Korrketure durch Sammelkeslinse\\
Kurzischtigkeit: Bild weiter Gegenstandesr vor Netrtzthauuta $\rightarrow$Korrektur durch Zerstreuungslinse\\
Stabsichtige Auge (AstigmaAutismus): abnormale Hornhautverkrümmung $\rightarrow$Korrektur durch Zylinderlinsen\\
\\Sehwinkel/räumliche Auflösung des Auges: $\varepsilon_0^{min} \approx 1" \Rightarrow \Delta x_{\text{min}} = S_0\cdot\varepsilon_o^{min} \approx 70 \mu m$\\
%TODO Die Lupe, Das Fernrohr VL17
Mikroskop: $V_{\text{Mikroskop}} = \frac{(l-f_e)\cdot L_d}{d_0\cdot f_e} = \beta_{\text{Objektiv}}\cdot V_{\text{Okular}}$\\
Vergrößerung Okular: $V_{\text{Okular}} = \frac{L_d}{F_e}$ $L_d$ = deutliche Sehweite des Mensche, ca 250mm\\
Auflösungsgrenze bei ca 1000-facher Vergrößerung\\
%TODO Dicke Linsen S16 VL 17, Fresnel Linse

\subsection{Abbildungsfehler (Abberationen} %TODO ausarbeiten VL17,evtl mit Bildern
\subsubsection{Schärfefehler}
Sphärische Abberationen: Groößerer EInfallswinkel am Rand $\rightarrow$ Zerstreuungskreis (Kaustik)\\
Koma
Astigmatismus
\subsubsection{Lagefehler}
Bildfeldwölbung, Verzeichnung
\subsubsection{Farbefehler/Chromatische Abberationen}
Farblängsfehler, Farbquerfehler $\rightarrow$ Dispersion\\

\subsection{Welleneigenschaft des Lichts}
$W_{\text{Welle}} = W_{\text{el}}+W_{\text{magn}} = \frac{1}{2}\cdot\varepsilon_0\cdot E^2 + \frac{1}{2\mu_0}\cdot B^2$\\
mit $E = \frac{1}{\sqrt{\mu_0\varepsilon_0}}\cdot B = c\cdot B \rightarrow W_{\text{Welle}} = \varepsilon_0E^2 = \frac{B^2}{\mu_0}$\\


 







	
\end{multicols*}
\end{document}


