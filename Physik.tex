% .:: Laden der LaTeX4EI Formelsammlungsvorlage
\documentclass[fs, footer]{latex4ei}
\usepackage[european]{circuitikz}

\usepackage{multirow}
\usepackage{latexnew}

% Ausgegraut zum Abschreiben:
%\definecolor{grey}{rgb}{0.6,0.6,0.6}
%\color{grey}

\renewcommand{\thesection}{\Roman{section}}
%\renewcommand{\thesubsection}{\thesection.\Roman{subsection}}
%\renewcommand{\thesubsubsection}{\thesection.\thesubsection.\Roman{subsubsection}}

\iffalse
\fancyfoot[R]{Created \today \ at \thistime \qquad \thepage}
\fancyfoot[L]{Homepage: \href{www.latex4ei.de}{www.latex4ei.de} -- please report misstakes \emph{immediately}.}
\fancyfoot[C]{by Emanuel Regnath, contact \emph{\href{mailto:emanuel.regnath@tum.de}{emanuel.regnath@tum.de}}}
\fi

% Dokumentbeginn
% ======================================================================
\begin{document}


% Aufteilung in Spalten
\vspace*{-10mm} %TODO Später anpassen
\begin{multicols*}{4}


	\fstitle{Physik}

\iffalse
	\emphbox{
	\textbf{Wichtiger Hinweis}
	\\ Diese Formelsammlung ist noch in der Entwicklung und nicht prüfungstauglich ! \\ Allerdings würden wir uns über Unterstützung freuen das zu ändern. Wer Lust hat kann uns über das Kontaktformular auf www.latex4ei.de erreichen.
	}\fi
% ===============================================================================================
\section{Physikalische Größen und Einheiten}
%Kinematik; 
%Dynamik für Punktmassen: Kräfte & Newtonsche Gesetze; 
%Arbeit, Energie, Leistung
%Stöße zwischen Punktmassen
%Dynamik des starren Körpers

%7 SI Basiseinheiten: Sekunde, Meter, Kilogramm, Ampere, Candela, Kelvin, Mol\\ Abgeleitete Einheiten
\subsection{Messgenauigkeit und Messfehler}
Systematischer Fehler: Abw. einer Messung von ihrem Erwartungswert\\
Statistischer Fehler: Entstehung durch zufällige Abweichungen\\
Arithmetischer Mittelwert: $\ol x = \fr{1}{n}\sum_{i=1}^n x_i$\\
Standardabweichung: $s = \sigma = \sqrt{\fr{1}{n-1}\sum_{i=1}^n (x_i-\ol x)^2}$\\
Standardabweichung mit TR: $s_{\text{Rechner}} = \sqrt{ \fr{\sum_{i=1}^n x_i^2 - \frac{1}{n} (\sum_{i=1}^n x_i)}{n-1} }$\\
Normalverteilung/Gauß-Funktion: $g(x) = \fr{1}{\sigma \sqrt{2\pi}}\exp(-\fr{(x-\ol x)^2}{2\sigma^2})$\\
Näherungsweise gilt: 
\begin{itemize}
\item 68(95)[99.8]\% aller Messwerte haben eine Abweichung $<$ $\pm 1(2)[3]\sigma$ vom Mittelwert.
%\item 95\% aller Messwerte haben eine Abweichung < $\pm 2\sigma$ vom Mittelwert.
%\item 99,8\% aller Messwerte haben eine Abweichung < $\pm 3\sigma$ vom Mittelwert.
\end{itemize}


\subsection{Konstanten}
%TODO Konstanten vervollständigen %TODO
$ \varepsilon_0 = 8.85\cdot 10^{12} \fr{C^2}{Nm^2}$\\
$ c_0 = 299 792 458 \frac{m}{s} \approx 3\cdot 10^8 \frac{m}{s}$\\
Boltzmannkonstante $k_B = \frac{R}{N_{Av}} = 1.381\cdot 10^{-23}$\\
Plank'sches Wirkungsquantum h $= 6.626\cdot 10^{-34}Js = 4.136\cdot 10^{-15}eVs$\\
Gaskonstante $R = C_{\text{p(mol)}} - C_{\text{v(mol)}} = 8.314 \frac{J}{mol\cdot K}$\\

%TODO Fehlerfortpflanzung EHER NICHT %TODO
%TODO Additionstheoreme EHER NICHT %TODO

\subsection{Trigonometrische Funktionen}

$\begin{array}{c|c|c|c|c|c|c|c|c}
x & 0 & \pi / 6 & \pi / 4 & \pi / 3 & \pi / 2 & \pi & \frac{3}{2}\pi & 2 \pi \\ \hline
\sin & 0 & \frac{1}{2} & \frac{1}{\sqrt{2}} & \frac{\sqrt 3}{2} & 1 & 0 & -1 & 0 \\
\cos & 1 & \frac{\sqrt 3}{2} & \frac{1}{\sqrt 2} & \frac{1}{2} & 0 & -1 & 0 & 1 \\     
\tan & 0 & \frac{\sqrt{3}}{3}&	1				 &	\sqrt{3} & \infty & 0 & - \infty & 0\\
\end{array}$ 

\iffalse
\begin{tabular}{l  l} 
		Additionstheoreme &	Stammfunktionen \\
	$\cosh x \,\; + \sinh x \,\,= e^{x}$ & $\int \sinh x \, dx = \cosh x + C$\\
	$\sinh({\rm arcosh}(x)) = \sqrt{x^2 - 1}$ & $\int \cosh x \, dx = \sinh x + C $\\
	$\cosh({\rm arsinh}(x)) = \sqrt{x^2 + 1}$ \\
	
 	$\cos (x - \frac{\pi}{2}) = \sin x$ & $\int x \cos(x) \diff x = \cos(x) + x \sin(x)$\\
 	
 	 $\sin (x + \frac{\pi}{2}) = \cos x$ & $\int x \sin(x) \diff x = \sin(x) - x \cos(x)$\\
 	
 	$\sin 2x = 2 \sin x \cos x $  & $\int \sin^2(x) \diff x = \frac12 \bigl(x - \sin(x)\cos(x) \bigr)$\\
     
 	$\cos 2x = 2\cos^2 x - 1$  & $\int \cos^2(x) \diff x = \frac12 \bigl(x + \sin(x)\cos(x) \bigr)$\\

 	$\sin(x) = \tan(x)\cos(x)$ & $\int \cos(x)\sin(x) = -\frac12 \cos^2(x)$ \\
\end{tabular}
\fi

\subsection{Quadratische Gleichung}
$x_1,_2 = \frac{-b \pm \sqrt{b^2 - 4ac}}{2a}$ oder $P_{\text{1,2}} = \frac{-p}{2}\cdot\sqrt{\frac{p}{2}^2 - q}$\\




\section{Klassische Mechanik}
%Kinematik
%Dynamik für Punktmassen: Kräfte & Newtonsche Gesetze
%Kräfte, Arbeit, Energie, Leistung
%Stöße zwischen Punktmassen
%Dynamik des starren Körpers



\subsection{Kinematik}
momentane Geschwindigkeit: $v = \dot r$\\
mittlere Geschwindigkeit: $v_m = \fr{\Delta r}{\Delta t}$\\
\subsubsection{Galilei Transformation}
Gilt nur für $v<<c$\\
$x' = x - ut$ und $t' = t$ mit der Geschwindigkeit $u$ des bewegten Systems $\ra \dt x = \dt{x'} + u$\\
Transformation erleichtert Bezugssystem mit konstanter Geschwindigkeit
$\rightarrow$ Berechnung im Schwerpunktsystem
\subsubsection{Eindimensionale Bewegungen}
Mittlere Beschleunigung: $a = \dt v$\\
Gleichförmige, geradlinige Bewegung: $x(t) = v_0t+c$\\
Gleichförmig beschleunigte Bewegung: $x(t) = \fr{1}{2}a_0t^2+v_0t+x_0$\\
Momentane Geschwindigkeit: $v = \dt r$\\
\subsubsection{Zweidimensionale Bewegungen}
Unabhängige Bewegungen in den einzelnen Raumrichtungen\\
Schiefer Wurf:
Berechnung von z(x) durch Eliminieren von t: \\$x(t) = v_{0x}t \Rightarrow t = \fr{x}{v_{0x}}$\\
$z(x) = -\fr{1}{2}g(\fr{{x}}{v_{0x}})^2 + \fr{v_{0z}}{v_{0x}}x = -\fr{g}{2v^2_{x0}}x^2 + tan\theta x$\\
\subsection{Dynamik für Punktmassen}
\subsubsection{Schiefe Ebene}
Gewichtskraft: $F_G = mg$\\
Normalkraft: $F_N = mg\cos \alpha$\\
Hangabtriebskraft: $F_H = F_A = mg\sin \alpha$\\
Reibung: Körper steht, falls $F_{\text{Haft}} = F_{\text{Hang}}$\\
kritischer Neigungswinkel: $tan \alpha = \mu_h$\\
\subsubsection{Kreisbewegung}
Winkel $\phi = \fr{s}{r}$, 	Bogenlänge $s$, Radius $r$, Umlaufdauer $T$\\
Kreisfrequenz $\omega = \dt \phi = \fr{2\pi}{T} = 2\pi f$, mit	Frequenz $f$\\
Krummlinige Bewegung: $\ \v a = \dt {\v v} = \v a_t + \v a_{zp}$,	Tangentialbeschl. $a_t$\\

\subsection{Kräfte, Arbeit, Energie, Leistung}
\subsubsection{Kraft}
Für $m_t \neq$ const: $\v F = m_t\frac{d}{dt}\v v +\v v\frac{d}{dt},_t$\\ %TODO DRINNEN LASSEN ODER RAUS NEHMEN? %TODO
Kräfte werden vektoriell addiert: $\v F_{ges} = \sum_{i=1}^n \v F_i$\\
Gravitationskraft: $F_G = -G \fr{m_1m_2}{r_{12}^2}$, mit $G = 6,67\cdot 10^{-11} \fr{Nm^2}{kg}$\\
Zentripetalkraft: $F_Z = m \fr{v^2}{r} = m\omega^2 r$\\
Federkraft (Hooke'sches Gesetz): $\v F_F = -k\v x$\\
mittlere Kraft: $\abs{<\v F>} = \abs{\fr{\Delta p}{\Delta t}} = \abs{\fr{m(v_E - v_A)}{\Delta t}}$\\
Coulombkraft: $F = \fr{1}{4 \pi \varepsilon_0}\fr{Q_1Q_2}{r^2}$\\
Reibungskräfte allgemein: $\v F_R = \mu \v F$	,z.B. Haft-, Gleit- und Rollreibung\\
Körper beginnt zu rutschen, wenn $\mu_H >= tan \theta$\\
Luftwiderstand: $\v F_W = \frac{1}{2}\rho c_WAv^2$ mit $\rho$: Luftdichte\\
\subsubsection{Arbeit}
Generell: $W = \int_{r_1}^{r_2} F dr$ bzw. $W = Fs\cos \alpha$\\
Spannarbeit an einer Feder: $W = \fr{1}{2} k(x-x_a)^2 $\\
\subsubsection{Energie}
potentielle Energie: $E_{pot} = mgh$\\
kinetische Energie: $E_{kin} = \fr{1}{2}mv^2$\\
$\textbf{Energieerhaltung}$: Grundprinzip: $E_{vorher} = E_{nachher}$\\
Gesamte Rotationsenergie: $E_{rot} = \sum_{i=1}^N \fr{1}{2} \Delta m_ir_{i\perp}^2\omega ^2$\\
\subsubsection{Leistung}
$P = \dt W = \v F\v V = \dt E$\\

\subsection{Scheinkräfte}%TODO Evtl subsection Scheinkräfte VL5
Zentrifugalkraft $\v F_f = - \v F_z$, Kompensation zur Zentripetalkraft\\
Corioliskraft $\v F_c = m\v a_c = 2m\v v \times \v w$\\

\subsubsection{Stöße}
Impuls: $p = mv$, $F=\dot p$\\
\subsubsection{Inelastischer Stoß}
Massen bilden gemeinsame Masse: $v_1' = v_2' = v'$\\
\subsubsection{Elastischer Stoß}
Fall $m_1 = m_2$: $v_1' = v_2, v_2' = v_1$\\
Fall $m_1 = m_2, v_1 \neq 0, v_2 = 0$: $v_1' = 0, v_2' = v_1$\\
$\v v_{\text{1,end}} = \frac{1}{m_1+m_2}\left((m_1-m_2)\v v_{\text{1,anf}} + 2m_1\v v_{\text{2,anf}}\right)$\\

\subsubsection{Drehungen}
Drehmoment: $\v M = \v r \times \v F$\\
Drehimpuls: $\v L = \v r \times \v p$\\
%TODO $ \rightarrow \v L = mr^2 \v \omega \rightarrow \v L = J \v \omega $\\ %TODO fixen
Trägheitsmoment: $J = \sum_{i=1}^n m_iR_i^2 = \int_V r_\perp^2\rho dV$\\
Satz von Steiner: $J = J_S + Md^2$, Bei bel. Achse A: Summe vom $J_S$ der Rotation durch Schwerpunkt + $Md^2$ von Schwerpunkt um A\\ %TODO BESSER
$E_{kin}(\Delta m_i)=\frac{1}{2}\Delta m_iv_i^2=\frac{1}{2}\Delta m_ir_{i\perp}^2 \omega^2$\\
Gesamte Rotationsenergie: $E_{\text{rot}}=\lim_{N \rightarrow \infty} (\sum_i^N  \fr{1}{2}\Delta m_ir_{i\perp}^2 \omega^2)=\frac{1}{2} \omega ^2\int_Vr_\perp^2 dm$\\
Für ein Teichensystem: $J = \sum_{i}m_ir_{i\perp}^2 \Rightarrow E_{\text{rot}}=\fr{1}{2}J\omega^2$\\
\subsection{Dynamik des starren Körpers}
Massenschwerpunkt $\v R_s = \frac{1}{M}\sum_im_i\v r_i$\\ %TODO Massenmittelpunkt VL8
\subsubsection{Trägheitsmomente}
%Vollzylinder: $J = \fr{1}{2}h\pi \rho R^4$ 
%Hohlzylinder: $J = \fr{1}{2}h\pi \rho[R^4-(R-d)^4] \approx 2 \pi h\rho R^3d$\\
$\textbf{Drehachse ist Körperachse:}$\\
Vollzylinder: $J = \frac{1}{2}m_{\text{ges}}r^2$\\
Zylindermantel: $J = m_{\text{ges}}r^2$\\
Hohlzylinder: $J = \frac{1}{2}m_{\text{ges}}(r_1^2+r_2^2)$\\

$\textbf{Drehachse durch Mittelpunkt $\perp$ Körperachse:}$\\
Zylindermantel: $J = \frac{1}{2}m_{\text{ges}}r^2 + \frac{1}{12}m_{\text{ges}}l^2$\\
Vollzylinder: $J = \frac{1}{4}m_{\text{ges}}r^2 + \frac{1}{12}m_{\text{ges}}l^2$\\
Dünner Stab: $J = \frac{1}{12}m_{\text{ges}}l^2$, (Drehachse durch Mittelpunkt)\\
Dünner Stab:  $J = \frac{1}{3}m_{\text{ges}}l^2$, (Drehachse durch ein Ende)\\
Dünne Kugelschale: $J = \frac{2}{3}m_{\text{ges}}r^2$, (Drehachse durch Mittelpunkt)\\
Massive Kugel: $J = \frac{2}{5}m_{\text{ges}}r^2$, (Drehachse durch Mittelpunkt)\\
Massiver Quader: $J = \frac{1}{12}m_{\text{ges}}(a^2+b^2)$, (Drehachse durch Oberfläche)\\


Masse des Zylindermantel: $M \approx 2 \pi R h d \rho$\\
Energieerhalt. rollender Zylinder: $E_{\text{pot}} = E_{\text{kin,translation}} + E_{\text{rotation}} \rightarrow mgh = \fr{1}{2}mv_s^2 + \fr{1}{2}J\omega ^2; s = r\alpha, v = r\omega$\\
%TODO Tabelle Trägheitsmomente Demtröder VL9 S11 EINFUEGEN %TODO

%TODO evtl analogien Translation Rotation VL10 %TODO

\subsection{Planetenbewegung}
\begin{itemize}
\item[1] Keplersches Gesetz: Planetenbahnen sind Ellipsen um Stern in einem der beiden Brennpunkte\\
\item[2] Keplersches Gesetz: In gleicher Zeit wird die gleiche Flaeche an einer Bahn aufgespannt\\$\frac{dA}{dt} = \frac{1}{2} \v r \v v sin \alpha = \frac{1}{2}\v r \times \v v = fr{1}{2m}|\v L| \Rightarrow$ Drehimpuls ist zeitlich konstant\\
\item[3] Keplersches Gesetz: $\frac{T_1^2}{T_2^2} = \frac{a_1^3}{a_2^3}$ mit T: Umlaufzeit, a: Große Halbachse\\
\end{itemize}
%TODO evtl Spezialfall Kreisbahn EHER NICHT %TODO

\section{Wellenlehre und Optik}
%Schwingungen
%Wellen
%Geometrische Optik
%Lichtwellen
%Laser
\subsection{Schwingungen}
Erzwungen: Amplitude $A(\omega) = \frac{\frac{F_0}{m}}{\sqrt{(\omega_0^2 - \omega^2)^2+ (2\gamma\omega)^2}}$, mit Resonanzfrequenz $\omega_0$, Abklingkonstante $\gamma = \frac{2b}{m}$\\

Logarithmisches Dekrement $\Lambda = \ln\frac{x_m}{x_n}=\gamma\cdot T = \frac{2\pi\gamma}{\sqrt{\omega_0^2 - \gamma^2}}$ (Maß für Dämpfungsverhalten)\\
Dämpfungsgrad $D = \frac{\gamma}{\omega_0}$\\
Gütefaktor Q eines Oszillators: $Q = \frac{\omega_0}{2\gamma}$\\

Falls von Reibung dominiert: $A = \frac{F_0}{b\sqrt{\frac{k}{m}}}$\\
Überlagerung von Schwingungen: $x(t)=\sum_nx_n(t)=\sum_na_n\cos{\omega_nt+\delta_n}$\\
\subsection{Harmonische Schwingungen} $x(t) = A \cos(\omega_0t + \Phi)$\\, mit A = Amplitude; $\omega$ = Kreisfrequenz [rad/s]; f = Frequenz[1/s]; T = Schwingungsdauer 1/f; $\phi$ = Phasenkonstante;\\
\subsubsection{Federpendel}
$\omega ^2 = \frac{\text{Rücktreibende Kraft}}{\text{Einheitsmasse $\times$ Einheitsauslenkung}} = \frac{k}{m} \rightarrow \omega = \sqrt{\frac{k}{m}}$\\
$\omega= 2 \pi f \rightarrow f = \frac{1}{2\pi}\sqrt{\frac{k}{m}}$\\
Energiebilanz: $E_{\text{ges}} = E_{\text{pot}} + E_{\text{kin}} = \frac{1}{2}kx^2+\frac{1}{2}mv^2$\\
\subsubsection{Mathematisches Pendel} 
$F = -mg \sin \theta \approx -mg \theta$\\
Oft Kleinwinkelnäherung: Bis 15$^{\circ}$: Fehler $<$ 0.01\%\\ %TODO FEHLER FINDEN
$x = l\theta ; F=-\frac{mg}{l}x$\\
Hooke'sches Gesetz: Kraft proportional zur Auslenkung\\
$\omega = \sqrt{\fr{g}{l}}$
\subsubsection{Torsionsschwingungen}
Elastisches Rückstelldrehmoment $M = -D\theta = J\alpha$\\
mit Torsionskonstante D und $\alpha = \frac{d^2 \theta}{dt^2}$\\
$\ddot{\theta}+\frac{D}{J}\theta = 0 \Rightarrow \omega = \sqrt{\frac{D}{J}}$\\
\subsubsection{Gedämpfter harmonischer Oszillator}
Stoke'sche Reibungskraft: $F_R = -bv = -b\dot{x}$\\
Bewegungsgleichung: $\ddot{x} + 2\gamma \dot{x} + \omega_0^2x = 0$; mit 2$\gamma$ = $\frac{b}{m}$\\
Lösungsansatz mit Cosinus: $x = Ae^{-\gamma t} \cos(\omega 't)$ \\mit $\omega ' = \sqrt{\omega_0^2-\gamma^2}, \gamma = \frac{b}{2m}$, $\omega_0 = \sqrt{\frac{k}{m}}$\\
schwache Dämpfung: $\gamma < \omega_0 \rightarrow$ $x = Ae^{- \frac{t}{t_{\text{\tiny L}}}} \cos(\omega 't)$\\
aperiodischer Grenzfall: $\gamma = \omega_0 \rightarrow	\omega' = 0$\\
überkritische Dämpfung: $\gamma \gg \omega_0 \rightarrow \omega' = \sqrt{\omega_0^2 - \gamma^2}$ = img.\\ $\rightarrow$ Das System schwingt nicht, kehrt langsam in GGP zurück\\

$t_L$ = mittlere Lebensdauer, Zeit auf 1/e der Amplitude\\

\subsection{Wellen}
Polarisation in Materie: $\v P = \chi_e\varepsilon_0\v E$, mit $\chi_e$: Elektrische Suszeptibilität, Materialeigenschaft, i.A. komplex\\
Longitudinale Welle: Auslenkung in Ausbreitungsrichtung\\
Transversale Welle: Auslenkung normal zur Ausbreitungsrichtung\\
Geschwindigkeit Seilwelle: $\nu = \sqrt{\frac{F_T}{\mu}}$\\ 
$F_T$ = Zugspannung, $\mu$ spezifische Masse\\ %TODO evtl S11 VL11 mit Bild
%TODO Formeln Geschwindigkeit VL11
%TODO VL12 S12 oben %TODO GAR NICHT SO WICHTIG
Masse $m = \mu\cdot vt \rightarrow \mu = \frac{m}{vt}$\\
Elastizitätsmodul: $E = \frac{F / A}{\Delta l / l}$\\
Kompressionsmodul: $K = \frac{-p}{\Delta V / V}$\\
Ausbreitungsgeschw. $\nu_{\text{Transv.}} = \sqrt{\frac{F}{\mu}}$, $\nu_{\text{Longi.}} = \sqrt{\frac{E}{\rho}}$, $\nu_\text{l,Gas} = \sqrt{\frac{K}{\rho}}$\\ %in einer Zeile
%Ausbreitungsgeschwindigkeit Transversaal $\nu = \sqrt{\frac{F}{\mu}}$\\
%Ausbreitungsgeschwindigkeit Longitudinal $\nu_\text{l} = \sqrt{\frac{E}{\rho}}$\\
%Ausbreitungsgeschwindigkeit in Gasen: $\nu_\text{l,Gas} = \sqrt{\frac{K}{\rho}}$\\
Schwingungsenergie des Teilchens: $E = \frac{1}{2}kD_M^2$\\
$k = 4\pi ^2mf^2; E = 2\pi^2mf^2D_M^2$\\
$m = \rho V = \rho A v t; \Delta E = 2\pi^2 \rho A v \Delta tf^2D_M^2$\\
Durchschnittliche Leistung: $\ol{P} = \fr{\Delta E}{\Delta t} = 2\pi ^2 \rho a v f^2 D_M^2$\\
%TODO Overline oben definieren und stattdessen mean, RMS; Dach als peak definieren, etc %TODO NICHT WÄHREND PRÜFUNGSPHASE
Intensität: $ I = \fr{\ol{P}}{A} = 2\pi^2\rho v f^2D_M^2$\\
Intensität sphärische Welle: $I = \frac{\hat{P}}{q\pi r^2}$, $D_M \propto \frac{1}{r}$\\
%TODO Mathematische Beschreibung der Wellenausbreitung, Wellengleichung VL12 %TODO
Schallpegel $L = 10\log{\frac{I}{I_0}}$dB mit $I_0 = 10^{-12}\frac{W}{m^2}$, 1dB = 10Bel\\
%TODO evtl Schallpegeltabelle %TODO SICHER NICHT
%TODO Superpositionsprinzip VL13
\\
Reflexion bei elektr. Leitungen: $ r = \frac{Z_{\text{Last}}-Z_0}{Z_{\text{Last}}+Z_0}$\\
%TODO subsection Interferenz VL14, Überlagerung von Wellen unterschiedlicher Frequenz

\subsection{Geometrische Optik}
$f\cdot\lambda = c$\\
Vakuumlichtgeschwindigkeit $c_0=2,99792458\cdot 10^8 \frac{m}{s} = \frac{1}{\sqrt{\varepsilon_0 \mu_0}}$\\
%TODO was ist Licht? %TODO LOL NEIN
Energie Photonen: $h\cdot c$ mit Plank'schem Wirkungsquantum $h=6.626\cdot10^{-34}\frac{J}{s}$\\
Brechungsindex $n = \frac{c}{v}$\\
Dielektrizitätskonstante $\varepsilon = n^2 $\\
 Brechungsgesetz von Snellius: $\frac{\sin\theta_1}{\sin\theta_2} = \frac{v_1}{v_2}  = \frac{c/n_1}{c/n_2} = \frac{n_2}{n_1}$\\
%Snellius'sche Gesetz für bestimmte Winkel: $n_1\sin\theta_1 = n_2\sin\theta_2$\\
Licht bricht immer zum Medium mit dem höheren Index hin\\
$\textbf{Fermatsches Prinzip}$: Licht folgt dem Weg mit der kürzesten Laufzeit$: \frac{dt}{dx} = 0$; Optischer Weg: $\int_{\gamma}n$\\
%TODO evtl Fata Morgana
Optische Wand, parallelverschiebung um $\Delta d:$\\$d = t\cdot\sin(\alpha)\cdot\left[-\frac{\cos\alpha}{\sqrt{n^2- \sin ^2(\alpha}}\right]$\\
Totalreflexion: falls $\theta>\theta_g$: $\sin(\theta_g) = \frac{n_2}{n_1}$\\
%TODO Besser

%TODO evtl Propagationsverluste in der Glasfaser oder was auch immer
%TODO Bild vom Prisma einfügen VL15 S20
Brechungsindex $n$ ist frequenzabh. $n(\omega)$\\
Ausbreitungsgeschw. ist frequenzabh. $v(\omega)$ heißt Dispersion\\
Maxwell Relation: $n = \sqrt{\varepsilon_r\cdot \mu_r} \approx \sqrt{\varepsilon_r} = \sqrt{1+\chi_e}$\\ %Not sure
Elektrische Suszeptibilität: $\v P = N \cdot \v p$\\% %TODO war das nicht schon oben? N im text
%TODO Bewegungsgleichung erzwungene Schwingung VL16 S7
$x_0 = \frac{eE_0}{m(\omega_0^2-\omega^2}$\\
$\omega < \omega_0$: Auslenkung in Phase,
$\omega > \omega_0$: Auslenkung gegen Phase $F_{el}$\\
Dipolmoment: $|\v p(t)| = e\cdot x(t) = \left|\frac{e^2E_0\cdot sin(\omega t)}{m(\omega_0^2-}\omega^2\right|$\\ %TODO große Betragsstriche
$\chi _e(\omega) = \frac{Ne^2}{\varepsilon_0(\omega_0^2 - \omega^2)}$\\
Sellmeier Gleichung: $n^2(\lambda) = 1 + \frac{B_1\lambda^2}{\lambda^2-C_1}\frac{B_2\lambda^2}{\lambda^2-C_2}\frac{B_3\lambda^2}{\lambda^2-C_3}$\\ mit $B_i$ und $C_i$ (i $\in$ 1-3) Sellmeier Koeffizienten, experimentell ermittelt\\
Anormale Dispersion: n steigt mit $\lambda$\\
Normale Dispersion: n fällt mit $\lambda$\\
\subsection{Abbildung}
Entweder reales Bild oder virtuelles Bild (z.B. Spiegel)\\
%TODO Spiegelarten und Strahlengang evtl mit Bildern
Strahlenkonstruktion allgemein: $\frac{1}{b}= \frac{1}{f} - \frac{1}{g}$\\
fokale Länge f = $\frac{r}{2}$; Gegenstandsweite g; Bildweite b\\
Vorzeichen korrekt wählen %TODO Tabelle VL16 S18
%TODO Linsenarten mit Bildern evtl
Abbildungsmaßstab $V = \frac{B}{G}=\frac{-b}{g}$ mit V negativ: Bild umgekehrt\\
\subsubsection{Linsen}
Linsengleichung:
Gegenstandseite: $\frac{f}{g} = \frac{B}{B+G}$
Bildseite: $\frac{f}{b}= \frac{G}{G+B}$\\
Dünne vs dicke Linsen\\ %TODO
Reziproke Brennweite = Brechkraft $\rightarrow$ Einheit Dioptrie [D]= 1dpt = $\frac{1}{m}$\\
$g>f$: Reelles Bild;
$g<f$: Virtuelles Bild\\
Berechnung Brennweite: $\frac{1}{f}=(n-1)(\frac{1}{r_1}-\frac{1}{r_2})$\\
mit n = Brechungsindex der Linse, r Radien
%\\SKIZZE HIER SKIZZE HIER
%TODO SKIZZE
\subsubsection{Auge}
Weitsichtigkeit: Bild naher Gegenstände hinter Netzhaut\\
$\rightarrow$ Korrketur durch Sammellinse\\
Kurzischtigkeit: Bild weiter Gegenstände vor Netzhaut\\
$\rightarrow$Korrektur durch Zerstreuungslinse\\
Stabsichtige Auge (Astigmatismus): abnormale Hornhautverkrümmung $\rightarrow$Korrektur durch Zylinderlinsen\\
\\Sehwinkel/räumliche Auflösung des Auges: $\varepsilon_0^{min} \approx 1" \Rightarrow \Delta x_{\text{min}} = S_0\cdot\varepsilon_o^{min} \approx 70 \mu m$\\
%TODO Die Lupe, Das Fernrohr VL17
Mikroskop: $V_{\text{Mikroskop}} = \frac{(l-f_e)\cdot L_d}{d_0\cdot f_e} = \beta_{\text{Objektiv}}\cdot V_{\text{Okular}}$\\
Vergrößerung Okular: $V_{\text{Okular}} = \frac{L_d}{F_e}$\\
$L_d$ = deutliche Sehweite des Mensche, ca 250mm\\
Auflösungsgrenze bei ca 1000-facher Vergrößerung\\
%TODO Dicke Linsen S16 VL 17, Fresnel Linse

\subsection{Abbildungsfehler (Abberationen)} %TODO ausarbeiten VL17,evtl mit Bildern
%ausarbeiten VL17,evtl mit Bildern
\subsubsection{Schärfefehler}
Sphärische Abberationen; %: Größerer EInfallswinkel am Rand $\rightarrow$ Zerstreuungskreis (Kaustik)\\
Koma; 
Astigmatismus $\rightarrow$ Sinus ist nichtlinear\\
\subsubsection{Lagefehler}
Bildfeldwölbung; 
Verzeichnung $\rightarrow$ Sinus ist nichtlinear
\subsubsection{Farbfehler/Chromatische Abberationen}
Farblängsfehler; Farbquerfehler $\rightarrow$ Dispersion\\

\subsection{Welleneigenschaft des Lichts}
$W_{\text{Welle}} = W_{\text{el}}+W_{\text{magn}} = \frac{1}{2}\cdot\varepsilon_0\cdot E^2 + \frac{1}{2\mu_0}\cdot B^2$\\
mit $E = \frac{1}{\sqrt{\mu_0\varepsilon_0}}\cdot B = c\cdot B \rightarrow W_{\text{Welle}} = \varepsilon_0E^2 = \frac{B^2}{\mu_0}$\\
Permittivität $\varepsilon$: Gibt Durchlässigkeit eines Materials für elektrische Felder an\\
magnetische Permittivität (magn Leitfähigkeit) $\mu$: Gibt Durchlässigkeit von Materie für magnetische Felder\\
$\varepsilon = \varepsilon_r\varepsilon_0; \mu = \mu_r\mu_0$\\

Welleneigenschaften: 
Pointingvektor $\v S=\frac{1}{\mu_0}\cdot \v E \times \v B = \v E \times \v H$ \\
zeigt in Ausbreitungsrichtung, Betrag = Intensität der Strahlung\\
Intensität S = Energiedichte $\times$ Ausbreitungsgeschwindigkeit\\
$[S]  = \frac{W}{m^2}$\\
Lichtwellen sind transversale e-m-Wellen mit $\v E\perp \v B \perp\v k$, mit $\v k \parallel$ Achse\\
$\v E = \v E_0\cdot\cos(k\cdot z - \omega\cdot t - \Phi) = \v E_0\cdot \cos(\frac{2\pi}{\lambda}(z-c\cdot t)-\Phi)$\\
$\v B$ ist direkt mit $\v E$ verknüpft\\
Kohärenz = Gleiche Frequenz, Feste Phasendifferenz\\
Meiste Lichtquellen inkohärent, Ausnahme Laser\\
Bei inkohärentem Licht mittelt sich die Interferenz zu null\\

Leistung eines Dipols (max $10^{-10}$m): $P = \frac{2}{3}\cdot\frac{e^2\cdot\omega^4\cdot d^2}{4\pi\varepsilon_0\cdot c^3}$\\
mit $\omega^2\cdot d = a \equiv$ Beschleunigung bei zirkularer Frequenz $\omega$\\ 
Lebensdauer atomare Schwingung: 1ns bis 10ns\\
Kohärenzlänge (Wegstrecke in 1ns): 30cm\\
%TODO evtl Fresnel'scher Doppelspiegel, Interferometer, Entspiegelung, Newtonsche Ringe
Fabry-Perot-Interferometer: Wellenlängenauflösung: $\frac{\Delta \lambda}{\lambda} = \frac{n}{N}$\\
Huygens-Fresnel-Prinzip: jeder Raumpunkt ist Ausgangspunkt für eine neue Kugelwelle (Elementarwelle)\\

\subsubsection{Beugung am Einfachspalt}
Bedingung für Minima: $a\!\cdot\!\sin\theta = Z\cdot\lambda$, mit Z $\in 1,2,3,...$\\ %TODO besser
Bedingung für Maxima: $a\cdot\sin\theta = (Z+\frac{1}{2})\cdot\lambda$, mit Z $\in -\frac{1}{2},1,2,3,...$\\%TODO
\subsubsection{Beugung am Doppelspalt}
Gangunterschied $\Delta s = q\cdot\sin\alpha$\\
Konstruktive Interferenz für Richtungen mit: $\Delta s = Z \cdot \lambda$\\
Destruktive Interferenz für Richtungen mit: $\Delta s = (Z+\frac{1}{2})\lambda$\\
%TODO Strichgitter, Kreuzgittertier VL19 NEIN
%TODO evtl Fresnelsche Zonenplatte NEIN
\subsection{Mikroskop}
Auflösugnsvermögen Mikroskop mit Spalt b: $\Psi_{\text{min}} = \alpha = \arcsin\frac{\lambda}{b} \approx \frac{\lambda}{b}$ (Abbé Limit)\\
für runde Linse mit Durchmesser D: $D\cdot\sin\alpha = 1,22\frac{\lambda}{D}$; $\Psi_{\text{min}} = 1,22\frac{\lambda}{D}$\\
\subsubsection{Röntgenbeugung}
Bragg-Bedingung konstruktive Interferenz: $n\lambda = 2d\sin\theta $, n$\in \N $\\
\subsubsection{Polarisation von Licht}
e-m-Welle ist transversaal, also $\v E \perp \v k$ bzw $\v B\perp\v k$\\
linear polarisiert $\rightarrow$ E-Feld steht nur in eine Richtung\\
Richtung von $\v E$ ist die Polarisationsrichtung\\
von k,E aufgespannte Ebene: Polarisatonsebene\\
Emmissionsakt eines einzelnen Atoms i.d.R. polarisiert, ungeregelte Überlagerung $\rightarrow$ unpolarisiert\\
Zwei Polarisationen:
S (Senkrecht) oder P (Parallel) zur Einfallsebene\\
Einfallsebene: $\v k$ und $n$(dach) spannen Ebene auf\\
Polarisation muss gleich sein für Interferenz\\
linear, elliptisch und zirkular möglich\\
Superposition mehrerer ist möglich\\
$I' = I\cdot\cos ^2\alpha$\\





\section{Hydromechanik}
%Flüssigkeiten und Gase:
%Dichte
%Druck
%Oberflächenspannung

%Strömende Flüssigkeiten:
%Strömungen
%Reale Flüssigkeiten: Viskosität
%Strömung einer viskosen Flüssigkeit durch  ein Rohr


Dichte $\rho = \frac{m}{V}$\\
%TODO Dichte Süßwasser $ \approx \rho (T) = \rho_{\text{max}} -7 \cdot 10^{-3}(T-4)^2 $, T in °C\\
Normalkraft $\v F_N$ senkrecht zur Oberfläche A erzeugt Druck $p = \frac{F_N}{A}$\\
Schweredruck: $p_s = \rho_{\text{Fl}}\cdot h\cdot g$\\
%TODO Formel hydraulischer Lift? VL19
Kompressibilität $\kappa = -\frac{1}{p}\cdot \frac{\Delta V}{V} \rightarrow \frac{\delta V}{V}=-\kappa\cdot \delta p$\\
Kompressionsmodul K = $\frac{1}{\kappa}$\\
Schallgeschwindigkeit in Flüssigkeit: $v_0 = \sqrt{\frac{dp}{d\rho}} = \sqrt{\frac{1}{\rho\kappa}}$\\
Gewicht pro Volumen $\gamma = \rho g$ [N/$m^3$]\\
$\gamma_{\text{Wasser}} = 998\frac{kg}{m^3}\cdot 9.807\frac{m}{s^2} = 9790 \frac{N}{m^2}$\\
%TODO evtl Tabelle Dichten und Schallgeschwindigkeiten
Auftriebskraft: $F_A = \rho_{\text{Fl}}\cdot g\cdot V_K$ mit Volumen $V_K$\\
$V_{\text{Verdrängt }} = \frac{\rho_K}{\rho_{Fl}}\cdot V_K$; Einsinken bis $m_v = m_k$\\
$\rho_k < $ (=) [$>$]$ \rho_{\text{Fl}}$: Körper schwimmt (schwebt) [sinkt]\\ %EINE ZEILE
%$\rho_k < \rho_{\text{Fl}}$: Körper schwimmt\\
%$\rho_k = \rho_{\text{Fl}}$: Körper schwebt\\
%$\rho_k > \rho_{\text{Fl}}$: Körper sinkt\\
\\
Oberflächenspannung $\gamma = \frac{F}{L} = \frac{dE}{dA}$ (auch $\sigma$)\\
$\sigma_{\text{Wasser}} =$ 0.073 N/m\\
Kapillarspannung $p_{\text{kap}} = \sigma(\frac{1}{r_1}+\frac{1}{r_2})$\\
kreisrunde Kapillare: $p_{\text{kap}} = \frac{2\sigma}{r}\cos (\phi)$, mit Kontaktwinkel $\phi$ $\Leftrightarrow p_S = \rho\cdot g\cdot h_{\text{kap}}$\\
%TODO Steighöhe $h_{\text{kap}} = \frac{2\sigma}{\rho gr}\cos(\phi}$\\
%TODO Bild benetzend oder nicht

\subsection{Strömungen}
$\textbf{laminare Strömung:}$ kleine Geschwindigkeiten, große innere Reibung, geringe Reibung mit Wänden\\
$\textbf{turbulente Strömung:}$ große Geschwindigkeiten, geringe innere Reibung, hohe Reibung mit Wänden\\

Kontinuitätsgleichung für inkompressible Flüssigkeit: $A_1\cdot v_1 = A_2\cdot v_2$, mit Querschnittsfläche A\\
Volumenstrom $\dot V = \frac{dV}{dt} = A\cdot v$ ist konstant\\
%TODO evtl mit Masse des Volumenelements
Bernoulligleichung: $p + \rho\cdot g\cdot h + \frac{1}{2}\cdot \rho\cdot v^2 =$ const., mit geodätischer Höhe h\\
Hydrodynamisches Paradoxon: Gleichgewicht bei $mg = \iffalse $p_{\text{atm}}\cdot A - p_1\cdot A =$ \fi
\frac{1}{2} \rho  v^2  A$\\ 
ideale Gasgleichung: $ \fr{\rho _0}{P_0} = \fr{M}{RT}$\\
Luftdruck: $p(h) = 1013hPa\cdot\exp{-\frac{h}{h_s}}$, $h_s = \frac{RT}{Mg} = 8428m$\\

Viskosität: Lineares Geschwindigkeitsprofil: $F = \eta\cdot A\cdot\frac{v}{z}$ mit Abstand z\\
Viskosität $\eta$ ist stark Temperaturabhängig\\

Strömung einer viskosen Flüssigkeit durch ein Rohr: $v(r) = \frac{p_1-p_2}{4\cdot\eta\cdot l}\cdot(R^2-r^2)$, v steigt parabelförmig zur Mitte hin an\\
Volumenstrom: %$ \dot V =$ $\frac{\pi\cdot (p_1-p_2)}{2\cdot\eta\cdot l}\cdot\lfloor R^2\int_0^R r\cdot dr - \int_0^Rr^3\cdot dr\rfloor =$
%$\frac{\pi\cdot(p_1-p_2)}{2\cdot\eta\cdot l}\cdot \lfloor \frac{R^4}{2}-\frac{R^4}{4}\rfloor $\\ $\rightarrow$ 
Gesetz von Hagen-Poiseuille: $\dot V  = \frac{\pi\cdot(p_1-p_2)}{8\cdot\eta\cdot l}\cdot R^4$ (für laminare Strömung)\\

Reynolds Zahl: $R_e = \frac{v \cdot\rho \cdot L}{\eta}$ mit L: char. Länge/Durchm. des Körpers:\\ 
1) für $R_e >> 1$: Newtonsches Reibungsgesetz $F = c_W \cdot A \cdot \frac{\rho\cdot v^2}{2}$\\
2) für $R_e < 1$: Stokessches Reibungsgesetz $F = b\cdot v$\\
Rein laminare Strömung bei $R_e \leq 0.1$\\





\section{Thermodynamik}
Vielteilchensysteme $\rightarrow$ Mittelung\\

Wärmemenge Q bei Erwärmung: $Q = C_p\cdot (T_2 - T_1)$ mit $C_p =$ Wärmekapazität in $\frac{J}{K}$\\
Gaskonstante: $R = C_{\text{p(mol)}} - C_{\text{v(mol)}}$ bzw. $n R = C_p - C_p$, mit Wärmekapazität $C_p$ bei isobarer, $C_v$ bei isochorer Zustandsänderung\\
spezifische Wärmekapazität $c = \frac{C}{m} = \frac{\Delta Q}{\Delta T\cdot m}$, mit $\Delta Q$: Wärmezufuhr , $\Delta T$: Temperaturerhöhung, $m$: Masse des Körpers\\
%Wärmekapazität bei konstantem Druck (ideales Gas): $C_p = pV = n RT$\\

Zustandsgleichung des idealen Gases: $\rho\cdot V  = n \cdot R\cdot T = N\cdot k_B\cdot T$ mit $n$ Stoffmenge in Mol, R = 8.314 $\frac{J}{mol\cdot K}$, N Anzahl der Gasatome\\%$k_B = \frac{R}{N_{\text{Av}}} = 1.381\cdot 10^{-23}$ Boltzmannkonstante\\
\\Kinetische Gastheorie $pV = Nm\langle v_z^2 \rangle$\\
%TODO evtl VL 21 S6 oben
mittlere kin. Energie der Teilchen eines idealen Gases $\bar{E}_{\text{kin}} = \frac{3}{2}k_B T$\\ % Wir haben keine Freiheitsgrade eingeführt nur $3/2k_B$ angenommen
Gesamte Translationsenergie eines idealen Gases: $\frac{3}{2}\frac{RT}{M}$\\

\subsection{Hauptsätze der Thermodynamik}
\begin{itemize}
\item 0. Zwei Körper im thermischen Gleichgewicht zu einem dritten $\rightarrow$ Alle stehen untereinander im Gleichgewicht\\
\item 1. $\Delta U = \Delta Q + \Delta W \rightarrow $ Es gibt kein Perpetuum mobile erster Art - Maschine mit $>$100\% Wirkungsgrad\\
$\textbf{Verschiedene Möglichkeiten für Zustandsänderung:}$\\
%TODO BESSERE LISTE %TODO
a) Isobarer Prozess, $p =$ const. $\rightarrow$ im idealen Gas ist $C_p$ konstant $\Rightarrow Q_{12} = C_p \Delta T$\\
b) Isochorer Prozess: $V =$ const. $\rightarrow$ im idealen Gas ist $C_v$ konstant $\Rightarrow Q_{12} = \Delta U$\\
c) Isotherme Prozesse: $T =$ const. $\Rightarrow$ $W_{12} = -Q_{12}$ %Umsetzung der Wärmezufuhr in Arbeit $\Rightarrow W_{12} = \\$
$ =n RT \ln\frac{V_1}{V_2}$, Freiwerdende Wärme: $Q_{12} = -W_{12}$\\
d) Adiabatische Prozesse: $\Delta Q = 0$
%TODO Isochorer Prozess, Isothermer Prozess, adiabatischer Prozess verbessern

In differentieller Schreibweise: $\partial U = \partial W + \partial Q $\\
\item 2. Thermische Energie ist nicht in beliebigem Maße in andere Energiearten umwandelbar. $\eta$ < 1
\item 3. Nernst'sches Theorem: $\lim_{T \rightarrow 0} S(T) = 0$ (Entropie bei 0 K ist 0)\\
\\
\\
\end{itemize}
%TODO Formeln mit $\propto$
%TODO evtl VL22 S9 unten
%TODO INHALTSVERZEICHNIS: Grundlagen; Das ideale Gas; Zustandsänderungen: Hauptsätze der Thermodynamik; reversible und irreversible Prozesse, Entropie; thermodynamische Maschinen
%TODO Adiabatengleichung
%\subsection{Zustandsänderungen: Hauptsätze der Thermodynamik II}
\subsection{Zustandsänderungen, Thermodynamische Systeme}
Adiabatengleichung $p\cdot V^\kappa =$ const\\, mit $\kappa$ Adiabatenexponent ($\frac{C_p}{C_v}$)\\
Carnotscher Kreisprozess: Idee der Wärmekraftmaschine\\
Wirkungsgrad $\eta = \frac{\vert W\vert}{Q_{12}}$, $\eta_{\text{Carnot}} = \frac{\vert - n R(T_2-T_1)\cdot\ln(\frac{V_2}{V_1})\vert}{n RT_2\cdot\ln(\frac{V_2}{V_1})} = \frac{T_2-T_1}{T_2} < 1$\\
%\subsection{Zustandsänderungen, Thermodynamische Systeme}
\subsection{Reversible und irreversible Prozesse}
Reversibler Prozess: z.B. Carnot- oder Stirling- Motor %$\rightarrow$ Wärmekraftmaschine, die durch abwechselnde Kompression und Expansion eines Gases als Wärmepumpe/Kältemaschine (bzw. Motor bei externer Wärmeeinwirkung) genutzt werden kann.\\
Irreversibler Prozess $\eta_{\text{irreversibel}} < \eta_{\text{Carnot}}$: Es kann nicht mehr in den Ausganszustand zurückgegangen werden\\

Entropie S: $\partial S = \frac{dQ_{\text{rev}}}{T} \qquad \Delta S = \int \frac{dQ_{\text{rev}}}{T}$\\
Für ideales Gas: $\Delta S = n\cdot C_v\cdot\ln\frac{T_2}{T_1} + n\cdot R\cdot\ln\frac{V_2}{V_1}$\\
\subsubsection{Das reale Gas}
1. Endliche Ausdehnung der Moleküle: $V_{\text{real}} = V_{\text{ideal}} + nb$ mit b= Eigenvolumen von 1 Mol\\
2. Anziehung: van-der-Waals-Kraft: $p_{\text{real}} = p_{\text{ideal}} - a(\frac{n}{V^2}) \rightarrow$ Druckreduktion\\
mit Materialkonstante a, welche die vdW-Kräfte berücksichtigt\\

Van-der-Waals-Gasgleichung: $(p+a\frac{n^2}{V^2})\cdot(V-nb) = n\cdot R\cdot T$\\
\\
Wärmeleitung: $\dot Q = \frac{dQ}{dt} = - \lambda  A \frac{dT}{dx}$\\


\vspace{-8mm}
\section{Quantenmechanik}
%Welle-Teilchen-Dualismus
%Wellenfunktion
%Unschärferelation
%Schrödingergleichung
%Lösung der Schrödingergleichung für verschiedene Potentiale

Wärmestrahlung: Gesetz von Stefan und Boltzmann: 
Strahlungsleistung Schwarzkörper $P_s$ eines schwarzen Körpers = $ P_S = \sigma\cdot A\cdot T^4 $\\
Stefan-Boltzmann-Konstante: $\sigma = 5.670 \cdot 10^{-8}\frac{W}{m^2\cdot K^4}$\\
effektiv abgestrahlte Leistung: $\Delta P_s = \sigma\cdot A (T_1^4-T_2^4)$\\

Für nichtideale Körper $\Delta P_s = \varepsilon\sigma\cdot A (T_1^4-T_2^4)$\\
Wien'scher Verschiebungssatz: $\lambda_{max} = \frac{2897,8\mu\cdot K}{T}$\\

\subsection{Welle-Teilchen-Dualismus}
Der Photoeffekt: Metallplatte entlädt sich durch Beleuchtung mit kurzwelligem Licht\\
Um ein Elektron abzulösen ist Austrittsarbeit $W_A = E_{\text{ph}} - E_{\text{kin}}$ nötig\\
%Energie eines Photons: $E_{ph} = h\cdot f$ mit f = Frequenz des Lichts, h = Plank'sches Wirkungsquantum = $6.626\cdot 10^{-34}Js = 4.136\cdot 10^{-15}eVs$\\

Teilchen haben Welleneigenschaften\\
de Broglie-Wellenlänge $p = \frac{h}{\lambda} \Rightarrow \lambda = \frac{h}{p}$\\
Materialwellen: $\hbar = \frac{h}{2\pi} \Rightarrow p = \hbar 2\pi \frac{1}{\lambda}$\\
aus der Wellenmechanik: $2\pi\frac{1}{\lambda} = k \rightarrow \v p = \hbar \v k$\\
Impuls des Teilchens wird mit der Wellenzahl verknüpft\\

\subsection{Wellenfunktion}
Ansatz: eben Welle für ein Teilchen mit Masse $m_0$, das sich mit der Geschwindigkeit $v$ bewegt: $E = \hbar\omega$, $p =\hbar k$ $\rightarrow \psi(x,t) = Ce^{i(\omega t-kx)}$\\
$\psi(x,t) = Ce^{i(\frac{E}{\hbar}t - \frac{p}{h}x)}$ Die Phase ist das Argument des Imaginärteils der Exponentialfunktion!\\
Phasengeschwindigkeit ist die Geschwindigkeit, bei der die zeitliche Änderung der Phase gleich 0 ist:
 $\frac{d}{dt}(\omega t - kx) = 0 \Rightarrow \omega - kv_{\text{ph}} = 0 \Rightarrow v_{\text{ph}} = \frac{\omega}{k}$\\
%Photonen: $\v_{\text{ph}} = \frac{\omega}{k}$ 
%$= \frac{w\pi f}{\frac{2\pi}{\lambda}} = \frac{E}{p} = \frac{\frac{1}{2}mv_r^2}{mv_r} = \frac{1}{2}v_r $\\ $Massebehaftete Teilchen, nicht relativistisch, S25 VL23\\
Phasengeschwindigkeit $V_{\text{ph}}  \leftrightarrow$ Ausbreitungsgeschwindigkeit $\frac{1}{2}v_T$\\
$y(x,t) = 2y_0 \sin(kx-\omega t)\cos(\Delta kx - \Delta \omega t)$\\
\subsection{Unschärferelation}
Ansatz: $\psi(x,t)  \int_{k_o - \frac{\Delta k}{2}}^{k_0 + \frac{\Delta k}{2}} C(k)e^{i(\omega t - kx)} dk$
Lösung: $\psi(x,t) = 2C \frac{\sin(u\frac{\Delta k}{2}}{u}$\\
Teil der Lösung: $u = ( (\frac{d\omega}{dk})_{k0}\cdot t-x$ mit $\nu_{gr} = (\frac{d\omega}{dk}_{k0}$\\
$\Delta x\cdot \Delta k = 2\pi$\\
Breite der Wellenfunktion $\Delta x$ bei $\Delta k$ mit $p = \hbar m$\\
$\Delta x\cdot \Delta p_x \geq \hbar$\\
Genauigkeit der Frequenzmessung hängt von der Lebensdauer des Zustandes ab: $\Delta \omega = \frac{1}{\tau}$\\
Zur Deutung von $\psi(x,t)$:
"Wahrscheinlichkeitsdichte" $|\psi(x,t)|^2$\\ $\psi(x,t)$ muss NORMIERT werden, da die Summe aller Wahrscheinlichkeiten zum Auftreten des Teilchens an allen Orten x und Zeiten t gleich 1 ist (100\% !)\\

Allgemein im Raum:$\iiint$\\

\subsection{Schrödingergleichung}
Erwin Schrödinger (1887 - 1961)\\

Räumliche und zeitliche Entwicklung von $\psi$ und damit der Wahrscheinlichkeit $W(\v x, t) = |\psi(\v x, t)|^2$\\
Muss DGL erster Ordnung sein (damit an $t_0$ durch Anfangsbedingung bestimmt), muss homogen sein, Lösungen sollten harmonische Wellen sein, damit man sie Superpositionieren kann (z.B. für Wellenpakete)\\

Ansatz:$\psi(x,t) = Ae^{i(kx-\omega t)}$\\
$\psi(x,t) = Ae^{\frac{i}{\hbar} (p_x x - E_{\text{kin}} t) }$ Ziel ist DGL für $\psi$\\S
Stationärer Fall: E hängt nicht von t ab: $\psi(x) = Ae^{ikx}$\\
$E_{\text{kin}} + E_{\text{pot}} = E$ $E_{\text{kin}} = \frac{p^2}{2m} = \frac{\hbar^2 k^2}{2m}+ E_{\text{pot}} = E$\\
Zweimaliges Differenzieren von $\psi$:\\
$\frac{\partial^2\psi(x)}{\partial x^2} = -k^2\psi(x)$\\

Stationäre Schrödingergl. in einer Dim.: $-\frac{\hbar^2}{2m}\frac{\partial^2\psi}{\partial x^2}+E_{\text{pot}} = E\psi(x)$\\

Verallgemeinerung auf 3 Dimensionen: 
$ - \frac{\hbar^2}{2m}\Delta \psi + E_{\text{pot}} \psi = E \psi$\\
mit Laplaceoperator $\Delta = \frac{\partial^2}{\partial x^2} + \frac{\partial^2}{\partial y^2} + \frac{\partial^2}{\partial z^2}$\\ und Hamiltonoperator $\hat H = - \frac{\hbar^2}{2m}\Delta + E_{\text{pot}}$\\

Zeitabhängigkeit : $\frac{d\psi(x,t)}{dt} = - \frac{i}{\hbar}E_{\text{kin}}\psi$\\
Von vorher: $\Rightarrow E_{\text{kin}} = \frac{\partial^2\psi}{\partial x^2}\cdot( - \frac{\hbar^2}{2m}$\\

Zeitabhängige Schrödingergleichung: $i\hbar \frac{\partial \psi}{\partial t}= \hat H \psi$\\
$\hat H \psi = E \psi$\\


\subsection{Lösung der Schrödingergleichung für verschiedene Potentiale}
%TODO Evtl S14
Für $E_{\text{pot}} = 0$: $\bar H \psi = i\hbar \frac{\partial \psi}{\partial t}$\\
Für unendlichen Potentialtop: $ - \frac{\hbar^2}{2m}\frac{\partial^2 \psi}{\partial x} + E_{\text{pot}}\psi = E\psi$\\
Allgemeiner Lösungsansatz: $\psi = Ae^{ikx} + Be^{-ikx}$\\
Mit Randbedinungen: $\frac{\partial ^2 \psi}{\partial x^2}$ definiert und $\psi = 0$ bei $x=0 und x=a$\\

Ergebnis: $0 = 2Ai\sin(ka)$, $ka = n\pi$; $n = 1,2,3,...$ und $\psi = 2Ai\sin(\frac{n\pi}{a}x)$\\

Wichtige Eigenschaft: Die Energien $E_n$ sind diskret\\

Coulombsches Gesetz: $E_{\text{pot}}(r) = \frac{Q^2}{4\pi\varepsilon_0}\cdot\frac{1}{r}$\\

\end{multicols*}
\end{document}


