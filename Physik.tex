% .:: Laden der LaTeX4EI Formelsammlungsvorlage
\documentclass[fs, footer]{latex4ei}
\usepackage[european]{circuitikz}

\usepackage{multirow}
\usepackage{latexnew}


% Dokumentbeginn
% ======================================================================
\begin{document}


% Aufteilung in Spalten
\vspace{-4mm}
\begin{multicols*}{4}
	\fstitle{Physik}

	\emphbox{
	\textbf{Wichtiger Hinweis}
	\\ Diese Formelsammlung ist noch in der Entwicklung und nicht prüfungstauglich ! \\ Allerdings würden wir uns über Unterstützung freuen das zu ändern. Wer Lust hat kann uns über das Kontaktformular auf www.latex4ei.de erreichen.
	}
% ===============================================================================================
\section{Messgenauigkeit und Messfehler}
Systematischer Fehler: Abweichung einer Messung von ihrem Erwartungswert\\
Statistischer Fehler: Entstehung durch zufällige positive bzw. negative Abweichungen\\
Arithmetischer Mittelwert: $\ol x = \fr{1}{n}\sum_{i=1}^n x_i$\\
Standartabweichung: $s = \sigma = \sqrt{\fr{1}{n-1}\sum_{i=1}^n (x_i-\ol x)^2}$\\
Normalverteilung/Gauß-Funktion: $g(x) = \fr{1}{\sigma \sqrt{2\pi}\exp(-\fr{(x-\ol x)^2}{2\sigma^2})}$\\
Näherungsweise gilt: 
\begin{itemize}
\item 68\% aller Messwerte haben eine Abweichung < $\pm \sigma$ vom Mittelwert.
\item 95\% aller Messwerte haben eine Abweichung < $\pm 2\sigma$ vom Mittelwert.
\item 99,8\% aller Messwerte haben eine Abweichung < $\pm 3\sigma$ vom Mittelwert.
\end{itemize}
%TODO Fehlerfortpflanzung
\section{Kinetik}
momentane Geschwindigkeit: $v = \dot r$\\
mittlere Geschwindigkeit: $v_m = \fr{\Delta r}{\Delta t}$\\
\subsection{Galilei Transformation}
Gilt nur für $v<<c$\\
$x' = x - ut$ und $t' = t$ mit der Geschwindigkeit $u$ des bewegten Systems $\ra \dt x = \dt{x'} + u$\\
\subsection{Schiefe Ebene}
Gewichtskraft: $F_G = mg$\\
Normalkraft: $F_N = mg\cos \alpha$\\
Hangabtriebskraft: $F_H = F_A = mg\sin \alpha$\\
Reibung: Körper steht, falls $F_{Haft} = F_{Hang}$\\
kritischer Neigungswinkel: $tan \alpha = \mu_h$\\

\section{Kraft}
Gravitationskraft: $F_G = -G \fr{m_1m_2}{r_{12}^2}$, mit $G = 6,67\cdot 10^{-11} \fr{Nm^2}{kg}$\\
Zentripetalkraft: $F_Z = m \fr{v^2}{r} = m\omega r$\\
Federkraft: $F_F = -kx$\\
mittlere Kraft: $\abs{<\v F>} = \abs{\fr{\Delta p}{\Delta t}} = \abs{\fr{m(v_E - v_A)}{\Delta t}}$\\
\section{Arbeit}
	$W = \int_{r_1}^{r_2} F dr$ bzw. $W = Fs\cos \alpha$\\

\section{Energie}
potentielle Energie: $E_{pot} = mgh$\\
kinetische Energie: $E_{kin} = \fr{1}{2}mv^2$\\
\subsection{Energieerhaltung}
Grundprinzip: $E_{vorher} = E_{nachher}$\\

\section{Stöße}
Impuls: $p = mv$, $F=\dot p$\\
\subsection{Inelastischer Stoß}
Massen bilden gemeinsame Masse: $v_1' = v_2' = v'$\\
\subsection{Elastischer Stoß}
Fall $m_1 = m_2$: $v_1' = v_2, v_2' = v_1$\\
Fall $m_1 = m_2, v_1 \neq 0, v_2 = 0$: $v_1' = 0, v_2' = v_1$\\

\section{Galilei-Transformation}
Transformation erleichtert Bezugssystem mit konstanter Geschwindigkeit
-> Berechnung im Schwerpunktsystem

\section{Drehungen}
Drehmoment: $\v M = \v r \times \v F$\\




	
\end{multicols*}
\end{document}


